% \iffalse
%<*driver>
%% File: chickenize.dtx by Arno Trautmann, mail: arno dot trautmann at gmx dot de
\iffalse
%</driver>
%<*readme>
This is the README file of the chickenize package. The package provides several commands and lua functions to manipulate the input or output tokens of any Lua(La)TeX document. It is /not/ intended for serious documents!

To produce the package files, run lualatex on chickinize.dtx, wich should result in the creaction of the following files:
  chickenize.pdf  (documentation)
  chickenize.tex  (plainTeX user interface)
  chickenize.sty  (LaTeX user interface) [may vanish]
  chickenize.lua  (Lua package code) [does the actual work]

You need an up-to-date TeX Live (2011, if possible) to use this package.

For any comments or suggestions, contact me:
arno dot trautmann at gmx dot de

Hope you have fun with this package!
%</readme>
%<*driver>
\fi
\def\nameofplainTeX{plain}
\ifx\fmtname\nameofplainTeX\else
  \expandafter\begingroup
\fi
\input docstrip.tex
\keepsilent
\let\MetaPrefix\relax
\preamble

EXPERIMENTAL CODE

Do not distribute this file without also distributing the
source files specified above.

Do not distribute a modified version of this file under the same name.

\endpreamble
\let\MetaPrefix\DoubleperCent
\postamble
\endpostamble
\askforoverwritefalse

\generate{\file{chickenize.sty}{\from{chickenize.dtx}{package}}}
\generate{\file{chickenize.tex}{\from{chickenize.dtx}{tex}}}
\def\MetaPrefix{-- }
\def\luapostamble{%
  \MetaPrefix^^J%
  \MetaPrefix\space End of File `\outFileName'.%
}
\def\currentpostamble{\luapostamble}%

\generate{\file{chickenize.lua}{\from{chickenize.dtx}{lua}}}
\nopreamble\nopostamble
\generate{\file{readme.markdown}{\from{chickenize.dtx}{readme}}}


\ifx\fmtname\nameofplainTeX
  \expandafter\endbatchfile
\else
  \expandafter\endgroup
\fi
\ProvidesFile{chickenize.sty}
  [v0.1 chickenize package]
\documentclass[english]{ltxdoc}
\usepackage{
  array,
  babel,
  booktabs,
  fontspec,
  graphicx,
  hyperref,
  scrpage2,
  tabu
}
\hypersetup{
  colorlinks=true,
  linkcolor=blue,
  filecolor=blue,
  urlcolor=blue
}
\cfoot{chicken \pagemark}
\ohead{}
\pagestyle{scrheadings}

\title{chickenize}
\author{Arno Trautmann\\ \href{mailto:arno.trautmann@gmx.de}{arno.trautmann@gmx.de}}
\begin{document}
\maketitle
\kern-6cm
{\hfill \parbox{6.3cm}{{\fontspec{TeX Gyre Chorus}{\large »\,}The Monthy Pythons, were they \TeX~users,\\\hspace*{.2em} could have written the chickenize macro.{\large \textit «}}\\
\hspace*{4cm}\small Paul Isambert}\kern-2.5cm
}

\kern4cm
\begin{abstract}
This is the package |chickenize|. It allows you to substitute or change the contents of a Lua\LaTeX\ document\footnote{The code is based on pure Lua\TeX\ features, so don't try to use it with any other \TeX\ flavour.}, but is actually only for fun. Please \emph{never} use any of the functionality of this package for a production document. The following table informs you shortly about your possibilities and provides links to the Lua functions. A \LaTeX\ and plain\TeX\ interface is also offered, see below.\\[2ex]
\begin{tabu}{lX[j]}
\toprule
function & \hfill effect \hfill\, \\
\midrule
\hyperref[sec:chickenize]{chickenize} & replaces every word with “chicken”\\
\hyperref[sec:colorstretch]{colorstretch} & shows grey boxes that depict the badness of a line\\
\hyperref[sec:leetspeak]{leetspeak} & translates every letter into the corresponding 1337 letter\\
\hyperref[sec:randomuclc]{randomuclc} & changes randomly between uppercase and lowercase\\
\hyperref[sec:randomfonts]{randomfonts} & changes the font randomly between every letter\\
\hyperref[sec:randomchars]{randomchars} & randomizes the whole input\\
\hyperref[sec:uppercasecolor]{uppercasecolor} & adds a color to every uppercase letter\\
\bottomrule
\end{tabu}
\vspace*{2ex}
\noindent If you have any suggestions or comments, just drop me a mail, I’ll be happy to get any response!
\end{abstract}
\tableofcontents
\section{How It Works}
We make use of Lua\TeX s callbacks, especially the |pre_linebreak_filter| and the |post_linebreak_filter|. Hooking a function into these, we can nearly arbitrarily change the contents of the document. If the changes should be on the input-side (replacing with |chicken|), one can use the |pre_linebreak_filter|. Hower, changes like inserting color are best made after the linebreak is finalized, so |post_linebreak_filter| is used for such things.

\section{How You Can Use It}
There are several ways to make use of this package. As all features are implemented on the Lua side, you can use these functions on their own. If you do so, please consult the corresponding subsections in the \hyperref{implementation}[sec:implementation] part, because there are some variables that can be adapted to your need.

You can use the following code inside a |\directlua| statement or in a |luacode| environment (or the corresponding thing in your format):
\begin{verbatim}
luatexbase.add_to_callback("pre_linebreak_filter",chickenize,"chickenize the input",1)
\end{verbatim}
Replace |"pre| by |"post| to register into the post linebreak filter. The second argument gives the function name; find a list of available functions below. You can give a label as you like in the third argument, and the last argument gives the order in which the functions in the callback are used. If you have no fancy stuff going on, you can safely use |1|.

If you don't want to mess with the Lua side (but please, try it, you'll learn much!), there is a \LaTeX- as well as a plain\TeX\ interface described in the next section. The commands may not always be on the latest code base – if anything does not work as expected, please tell me and I'll correct it.

\subsection{Commands – Document Wide}
You have a number of commands at your hand, each of which does some manipulation of the input or output. In fact, the code is easy and straightforward, but be careful, especially when combining things. Apply features step by step so your brain won't be damaged …

The effect of the commands can be influenced, not with arguments, but only via the |\chickenizesetup| described \hyperlink{adjustment}{below}.

\begin{description}
\def\command#1{\item[\ttfamily \color{blue} \textbackslash#1]}
\command{chickenize} Replaces every word of the input with the word “chicken”. Maybe sometime the replaced word can be changed, but up to now, it's only chicken. To be a bit less static, about every 10\textsuperscript{th} chicken is uppercase. However, the beginning of a sentence is not recognized automatically.\footnote{If you have a nice implementation idea, I'd love to include this!}
\command{uppercasecolor} Makes every uppercase character in the input colored. At the moment, the color is randomized over the full rgb scale, but that will be adjustable once options are well implemented.
\command{randomuclc} Changes every character of the input into its uppercase or lowercase variant. Well, guess what the “random” means …
\command{randomfonts} Changes the font randomly for every character. If no parameters are given, all fonts that have been loaded are used, especially including math fonts.
\command{leetspeak} Translates the input into 1337 speak. If you don't understand that, lern it, n00b.
\command{colorstretch} Inspired by Paul Isambert's code, this command prints boxes instead of lines. The greyness of the first (left-hand) box corresponds to the badness of the line, i.\,e. it is a measure for how much the space between words has been extended to get proper paragraph justification. The second box on the right-hand side shows the amount of stretching/shrinking when font expansion is used. Together the box greyness give you information about how well the overall greyness of the typeset page is.% You may specifiy the optional arguments |[(no)keeptext]| to display the text or delete it, also |[(no)colorexpansion]| controls wether or not the font expansion should be evaluated or not.  %% FIXME: No, no optional s**it. Use variables.

This functionality is actually the only really usefull implementation of this package …
\end{description}

\subsection{text-Versions}
The functions of this package might be much more useful if applied only to a short sequence of words or single words instead of the whole document or paragraph. Therefore, most of the above-mentioned commands have a |\text|-version that takes an argument. |\textrandomcolor{foo}| results in a colored |foo| while the rest of the document keeps its color. However, to achieve this effect, still the whole node list has to be traversed, so it may slow down your document, even if you use |\textrandomcolor| only once. Fortunately, the effect is very small and mostly negligible.\footnote{On a 500 pages text-only \LaTeX\ document the dilation is on the order of 10\% with |textrandomcolor|, but other manipulations can take much more time. However, you are not supposed to make such long documents with |chickenize|!}

Please don't fool around by mixing a |\text|-version with the non-|\text|-version. If you feel like and are not please with the result, it is up to \emph{you} to provide a stable and working solution.

\hypertarget{adjustment}{}
\section{How to Adjust It}
There are several ways to change the behaviour of |chickenize| and its macros. Most of the options are Lua variables and can be set using |\chickenizesetup|.\footnote{To be honest, this is just \textbackslash|let| to \textbackslash|directlua|. But having a separate macro just feels more natural. However, it might change in the future, so an abstraction layer is nice to have.} But be \emph{careful!} The argument of |\chickenizesetup| is parsed directly to Lua, therefore you are \emph{not} using a comma-separated key-value list, but uncorrelated Lua commands, e.\,g.: |\chickenizesetup{randomfontslower = 1| |randomfontsupper = 0}| instead of |\chickenizesetup{randomfontslower = 1,| |randomfontsupper = 0}|. Ok?

The following list tries to keep kind of track of the options and variables. There is no guarantee for this list, and if you find something that is missing or doesn't work as described here, please inform me!

\def\arg#1{\textcolor{red}{\ttfamily <#1> }}
\def\opt#1{\textcolor{blue}{\ttfamily #1}}
\begin{description}
\item[\opt{randomfontslower}, \opt{randomfontsupper} = \arg{int}] These two integer variables determine the span of fonts used for the font randomization. Just play with them a bit to find out what they are doing.
\item[\opt{chickenstring} = \arg{string}] The string that is printed when using |\chickenize|. So far, this does not really work, especially breaking into lines and hyphenation. Remember that this is Lua input, so a string must be given with quotation marks: |chickenstring = "foo bar"|.
\item[\opt{leettable} = \arg{table}] From this table, the substitution for 1337 is taken. If you want to add or change an entry, you have to provide the unicode numbers of the characters, e.\,g. |leettable[101] = 50| replaces every |e| (|101|) with the number |3| (|50|).
\item[\opt{uclcratio} = \arg{float}] Gives the fraction of uppercases to lowercases in the |\randomuclc| mode. A higher number (up to 1) gives more uppercase letters. Guess what a lower number does.
\item[\opt{randomcolor\_grey} = \arg{bool}] For a printer-friendly version, this offeres a grey scale instead of an rgb value for |\randomcolor|.
\item[\opt{Rgb\_lower}, \opt{rGb\_upper} = \arg{int}] To specify the color space that is used for |\randomcolor|, you can specify six values, the upper and lower value for each color. The uppercase letter in the variable denotes the color, so |rGb_upper| gives the upper value for green etc. Possible values are between |1| and |254|. If you enter anything outside this, your pdf will become invalid and break. For grey scale, use |grey_lower| and |grey_upper|, with values between |0| (black) and |1000| (white), included. Default is |0| to |900| to prevent white letters.
\item[\opt{keeptext} = \arg{bool}] This is for the |\colorstretch| command. If set to |true|, the text of your document will be kept. This way, it is easier to identify bad lines and the reason for the badness.
\item[\opt{colorexpansion} = \arg{bool}] If |true|, two bars are shown of which the second one denotes the font expansion. Only usefull if font expansion is used. (You \emph{do} use font expansion, do you?)
\end{description}

\DocInput{chickenize.dtx}
\end{document}
%</driver>
%<*tex>
% \fi
%\part{Implementation}
%\label{sec:implementation}
%\section{\TeX\ file}
%    \begin{macrocode}
\input{luatexbase.sty}
\directlua{dofile("chickenize.lua")}

\def\chickenize{
  \directlua{luatexbase.add_to_callback("pre_linebreak_filter",chickenize,"chickenize the input",1)}
}
\def\colorstretch{
  \directlua{luatexbase.add_to_callback("post_linebreak_filter",colorstretch,"show stretch and expansion",1)}
}
\def\leetspeak{
  \directlua{luatexbase.add_to_callback("post_linebreak_filter",leet,"transform input to 1337",1)}
}
\def\randomcolor{
  \directlua{luatexbase.add_to_callback("post_linebreak_filter",randomcolor,"random color",1)}
}
\def\randomfonts{
  \directlua{luatexbase.add_to_callback("post_linebreak_filter",randomfonts,"random fonts",1)}
}
\def\randomuclc{
  \directlua{luatexbase.add_to_callback("pre_linebreak_filter",randomuclc,"randomize uc/lc chars",1)}
}
\def\uppercasecolor{
  \directlua{luatexbase.add_to_callback("post_linebreak_filter",uppercasecolor,"color all uc chars",1)}
}

\newluatexattribute\randcolorattr
\def\textrandomcolor#1{%
\randomcolor%
\setluatexattribute\randcolorattr{42}#1%
\unsetluatexattribute\randcolorattr%
\directlua{randomcolor_onlytext=true}%
\gdef\textrandomcolor#1{%
\setluatexattribute\randcolorattr{42}#1%
\unsetluatexattribute\randcolorattr}
} %% to turn off automatic all-colorizing

%    \end{macrocode}
% \iffalse
%</tex>
%<*package>
% \fi
% \section{Preparation}
% Loading of packages and defition of constants. Will change somewhat when migrating to expl3 (?)
%    \begin{macrocode}
% \iffalse
%<*driver>
%% File: chickenize.dtx by Arno Trautmann, mail: arno dot trautmann at gmx dot de
\iffalse
%</driver>
%<*readme>
This is the README file that should contain some important information. So far I can only tell you to run the lualatex on the file chickenize.dtx to produce the four files
  chickenize.pdf  (documentation)
  chickenize.tex  (low-level commands; plainTeX)
  chickenize.sty  (LaTeX user interface)
  chickenize.lua  (Lua package code)

You need an up-to-date TeX Live (2011, if possible) to use this package.

For any comments or suggestions, contact me:
arno dot trautmann at gmx dot de

Hope you have fun with this!
%</readme>
%<*driver>
\fi
\def\nameofplainTeX{plain}
\ifx\fmtname\nameofplainTeX\else
  \expandafter\begingroup
\fi
\input docstrip.tex
\keepsilent
\let\MetaPrefix\relax
\preamble

EXPERIMENTAL CODE

Do not distribute this file without also distributing the
source files specified above.

Do not distribute a modified version of this file under the same name.

\endpreamble
\let\MetaPrefix\DoubleperCent
\postamble
\endpostamble
\askforoverwritefalse

\generate{\file{chickenize.sty}{\from{chickenize.dtx}{package}}}
\generate{\file{chickenize.tex}{\from{chickenize.dtx}{tex}}}
\def\MetaPrefix{-- }
\def\luapostamble{%
  \MetaPrefix^^J%
  \MetaPrefix\space End of File `\outFileName'.%
}
\def\currentpostamble{\luapostamble}%

\generate{\file{chickenize.lua}{\from{chickenize.dtx}{lua}}}
\nopreamble\nopostamble
\generate{\file{readme.markdown}{\from{chickenize.dtx}{readme}}}


\ifx\fmtname\nameofplainTeX
  \expandafter\endbatchfile
\else
  \expandafter\endgroup
\fi
\ProvidesFile{chickenize.sty}
  [v0.1 chickenize package]
\documentclass[english]{ltxdoc}
\usepackage{
  array,
  babel,
  booktabs,
  fontspec,
  graphicx,
  hyperref,
  tabu
}
\hypersetup{
  colorlinks=true,
  linkcolor=blue,
  filecolor=blue,
  urlcolor=blue
}

\setmainfont{TeX Gyre Pagella}

\title{chickenize}
\author{Arno Trautmann\\ \href{mailto:arno.trautmann@gmx.de}{arno.trautmann@gmx.de}}
\begin{document}
\maketitle
\kern-6cm
{\hfill \parbox{6.3cm}{{\fontspec{TeX Gyre Chorus}{\large »\,}The Monthy Pythons, were they \TeX~users,\\\hspace*{.2em} could have written the chickenize macro.{\large \textit «}}\\
\hspace*{4cm}\small Paul Isambert}\kern-2.5cm
}

\kern4cm
\begin{abstract}
This is the package |chickenize|. It allows you to substitute or change the contents of a Lua\LaTeX\ document\footnote{The code is based on pure Lua\TeX\ features, so don't try to use it with any other \TeX\ flavour.}, but is actually only for fun. Please \emph{never} use any of the functionality of this package for a production document. The following table informs you shortly about your possibilities and provides links to the Lua functions. A \LaTeX\ and plain\TeX\ interface is also offered, see below.\\[2ex]
\begin{tabu}{lX[j]}
\toprule
function & \hfill effect \hfill\, \\
\midrule
\hyperref[sec:chickenize]{chickenize} & replaces every word with “chicken”\\
\hyperref[sec:colorstretch]{colorstretch} & shows grey boxes that depict the badness of a line\\
\hyperref[sec:leetspeak]{leetspeak} & translates every letter into the corresponding 1337 letter\\
\hyperref[sec:randomuclc]{randomuclc} & changes randomly between uppercase and lowercase\\
\hyperref[sec:randomfonts]{randomfonts} & changes the font randomly between every letter\\
\hyperref[sec:randomchars]{randomchars} & randomizes the whole input\\
\hyperref[sec:uppercasecolor]{uppercasecolor} & adds a color to every uppercase letter\\
\bottomrule
\end{tabu}
\vspace*{2ex}
\noindent If you have any suggestions or comments, just drop me a mail, I’ll be happy to get any response!
\end{abstract}
\tableofcontents
\section{How It Works}
We make use of Lua\TeX s callbacks, especially the |pre_linebreak_filter| and the |post_linebreak_filter|. Hooking a function into these, we can nearly arbitrarily change the contents of the document. If the changes should be on the input-side (replacing with |chicken|), one can use the |pre_linebreak_filter|. Hower, changes like inserting color are best made after the linebreak is finalized, so |post_linebreak_filter| is used for such things.

\section{How You Can Use It}
There are several ways to make use of this package. As all features are implemented on the Lua side, you can use these functions on their own. If you do so, please consult the corresponding subsections in the \hyperref{implementation}[sec:implementation] part, because there are some variables that can be adapted to your need.

You can use the following code inside a |\directlua| statement or in a |luacode| environment (or the corresponding thing in your format):
\begin{verbatim}
luatexbase.add_to_callback("pre_linebreak_filter",chickenize,"chickenize the input",1)
\end{verbatim}
Replace |"pre| by |"post| to register into the post linebreak filter. The second argument gives the function name; find a list of available functions below. You can give a label as you like in the third argument, and the last argument gives the order in which the functions in the callback are used. If you have no fancy stuff going on, you can safely use |1|.

If you don't want to mess with the Lua side (but please, try it, you'll learn much!), there is a \LaTeX- as well as a plain\TeX\ interface described in the next section. The commands may not always be on the latest code base – if anything does not work as expected, please tell me and I'll correct it.

\subsection{Commands}
You have a number of commands at your hand, each of which does some manipulation of the input or output. In fact, the code is easy and straightforward, but be careful, especially when combining things. Apply features step by step so your brain won't be damaged …

Some commands have optional arguments that are \emph{only} available for \LaTeX. plain\TeX\ users are mostly capable of finding out how to change things themselfs, but if you are willing to wrap up the code for optional argument processing, don't hesitate sharing it with me ;)
\begin{description}
\item[chickenize] Replaces every word of the input with the word “chicken”. Maybe sometime the replaced word can be changed, but up to now, it's only chicken. To be a bit less static, about every 10\textsuperscript{th} chicken is uppercase. However, the beginning of a sentence is not recognized automatically.\footnote{If you have a nice implementation idea, I'd love to include this!}
\item[uppercasecolor] Makes every uppercase character in the input colored. At the moment, the color is randomized over the full rgb scale, but that will be adjustable once options are well implemented.
\item[randomuclc] Changes every character of the input into its uppercase or lowercase variant. Well, guess what the “random” means …
\item[randomfonts] Changes the font randomly for every character. If no parameters are given, all fonts that have been loaded are used, especially including math fonts.
\item[leetspeak] Translates the input into 1337 speak. If you don't understand that, lern it, n00b.
\item[colorstretch] Inspired by Paul Isambert's code, this command prints boxes instead of lines. The greyness of the first (left-hand) box corresponds to the badness of the line, i.\,e. it is a measure for how much the space between words has been extended to get proper paragraph justification. The second box on the right-hand side shows the amount of stretching/shrinking when font expansion is used. Together the box greyness give you information about how well the overall greyness of the typeset page is. You may specifiy the optional arguments |[(no)keeptext]| to display the text or delete it, also |[(no)colorexpansion]| controls wether or not the font expansion should be evaluated or not.

This functionality is actually the only really usefull implementation of this package …
\end{description}

\subsection{text-Versions}
The functions of this package might be much more useful if applied only to a short sequence of words or single words instead of the whole document or paragraph. Therefore, most of the above-mentioned commands have a |\text|-version that takes an argument. |\textrandomcolor{foo}| results in a colored |foo| while the rest of the document keeps its color. However, to achieve this effect, still the whole node list has to be traversed, so it may slow down your document, even if you use |\textrandomcolor| only once. Fortunately, the effect is very small and mostly negligible.\footnote{On a 500 pages text-only \LaTeX\ document the dilation is on the order of 10\% with |textrandomcolor|, but other manipulations can take much more time. However, you are not supposed to make such long documents with |chickenize|!}

Please don't fool around by mixing a |\text|-version with the non-|\text|-version. If you feel like and are not please with the result, it is up to \emph{you} to provide a stable and working solution.
\DocInput{chickenize.dtx}
\end{document}
%</driver>
%<*tex>
% \fi
%\part{Implementation}
%\label{sec:implementation}
%\section{\TeX\ file}
%    \begin{macrocode}
\input{luatexbase.sty}
\directlua{dofile("chickenize.lua")}

\def\chickenize{
  \directlua{luatexbase.add_to_callback("pre_linebreak_filter",chickenize,"chickenize the input",1)}
}
\def\colorstretch{
  \directlua{luatexbase.add_to_callback("post_linebreak_filter",colorstretch,"show stretch and expansion",1)}
}
\def\leetspeak{
  \directlua{luatexbase.add_to_callback("post_linebreak_filter",leet,"transform input to 1337",1)}
}
\def\randomcolor{
  \directlua{luatexbase.add_to_callback("post_linebreak_filter",randomcolor,"random color",1)}
}
\def\randomfonts{
  \directlua{luatexbase.add_to_callback("post_linebreak_filter",randomfonts,"random fonts",1)}
}
\def\randomuclc{
  \directlua{luatexbase.add_to_callback("pre_linebreak_filter",randomuclc,"randomize uc/lc chars",1)}
}
\def\uppercasecolor{
  \directlua{luatexbase.add_to_callback("post_linebreak_filter",uppercasecolor,"color all uc chars",1)}
}

\newluatexattribute\randcolorattr
\def\textrandomcolor#1{%
\randomcolor%
\setluatexattribute\randcolorattr{42}#1%
\unsetluatexattribute\randcolorattr%
\directlua{randomcolor_onlytext=true}%
\gdef\textrandomcolor#1{%
\setluatexattribute\randcolorattr{42}#1%
\unsetluatexattribute\randcolorattr}
} %% to turn off automatic all-colorizing

%    \end{macrocode}
% \iffalse
%</tex>
%<*package>
% \fi
% \section{Preparation}
% Loading of packages and defition of constants. Will change somewhat when migrating to expl3 (?)
%    \begin{macrocode}
% \iffalse
%<*driver>
%% File: chickenize.dtx by Arno Trautmann, mail: arno dot trautmann at gmx dot de
\iffalse
%</driver>
%<*readme>
This is the README file that should contain some important information. So far I can only tell you to run the lualatex on the file chickenize.dtx to produce the four files
  chickenize.pdf  (documentation)
  chickenize.tex  (low-level commands; plainTeX)
  chickenize.sty  (LaTeX user interface)
  chickenize.lua  (Lua package code)

You need an up-to-date TeX Live (2011, if possible) to use this package.

For any comments or suggestions, contact me:
arno dot trautmann at gmx dot de

Hope you have fun with this!
%</readme>
%<*driver>
\fi
\def\nameofplainTeX{plain}
\ifx\fmtname\nameofplainTeX\else
  \expandafter\begingroup
\fi
\input docstrip.tex
\keepsilent
\let\MetaPrefix\relax
\preamble

EXPERIMENTAL CODE

Do not distribute this file without also distributing the
source files specified above.

Do not distribute a modified version of this file under the same name.

\endpreamble
\let\MetaPrefix\DoubleperCent
\postamble
\endpostamble
\askforoverwritefalse

\generate{\file{chickenize.sty}{\from{chickenize.dtx}{package}}}
\generate{\file{chickenize.tex}{\from{chickenize.dtx}{tex}}}
\def\MetaPrefix{-- }
\def\luapostamble{%
  \MetaPrefix^^J%
  \MetaPrefix\space End of File `\outFileName'.%
}
\def\currentpostamble{\luapostamble}%

\generate{\file{chickenize.lua}{\from{chickenize.dtx}{lua}}}
\nopreamble\nopostamble
\generate{\file{readme.markdown}{\from{chickenize.dtx}{readme}}}


\ifx\fmtname\nameofplainTeX
  \expandafter\endbatchfile
\else
  \expandafter\endgroup
\fi
\ProvidesFile{chickenize.sty}
  [v0.1 chickenize package]
\documentclass[english]{ltxdoc}
\usepackage{
  array,
  babel,
  booktabs,
  fontspec,
  graphicx,
  hyperref,
  tabu
}
\hypersetup{
  colorlinks=true,
  linkcolor=blue,
  filecolor=blue,
  urlcolor=blue
}

\setmainfont{TeX Gyre Pagella}

\title{chickenize}
\author{Arno Trautmann\\ \href{mailto:arno.trautmann@gmx.de}{arno.trautmann@gmx.de}}
\begin{document}
\maketitle
\kern-6cm
{\hfill \parbox{6.3cm}{{\fontspec{TeX Gyre Chorus}{\large »\,}The Monthy Pythons, were they \TeX~users,\\\hspace*{.2em} could have written the chickenize macro.{\large \textit «}}\\
\hspace*{4cm}\small Paul Isambert}\kern-2.5cm
}

\kern4cm
\begin{abstract}
This is the package |chickenize|. It allows you to substitute or change the contents of a Lua\LaTeX\ document\footnote{The code is based on pure Lua\TeX\ features, so don't try to use it with any other \TeX\ flavour.}, but is actually only for fun. Please \emph{never} use any of the functionality of this package for a production document. The following table informs you shortly about your possibilities and provides links to the Lua functions. A \LaTeX\ and plain\TeX\ interface is also offered, see below.\\[2ex]
\begin{tabu}{lX[j]}
\toprule
function & \hfill effect \hfill\, \\
\midrule
\hyperref[sec:chickenize]{chickenize} & replaces every word with “chicken”\\
\hyperref[sec:colorstretch]{colorstretch} & shows grey boxes that depict the badness of a line\\
\hyperref[sec:leetspeak]{leetspeak} & translates every letter into the corresponding 1337 letter\\
\hyperref[sec:randomuclc]{randomuclc} & changes randomly between uppercase and lowercase\\
\hyperref[sec:randomfonts]{randomfonts} & changes the font randomly between every letter\\
\hyperref[sec:randomchars]{randomchars} & randomizes the whole input\\
\hyperref[sec:uppercasecolor]{uppercasecolor} & adds a color to every uppercase letter\\
\bottomrule
\end{tabu}
\vspace*{2ex}
\noindent If you have any suggestions or comments, just drop me a mail, I’ll be happy to get any response!
\end{abstract}
\tableofcontents
\section{How It Works}
We make use of Lua\TeX s callbacks, especially the |pre_linebreak_filter| and the |post_linebreak_filter|. Hooking a function into these, we can nearly arbitrarily change the contents of the document. If the changes should be on the input-side (replacing with |chicken|), one can use the |pre_linebreak_filter|. Hower, changes like inserting color are best made after the linebreak is finalized, so |post_linebreak_filter| is used for such things.

\section{How You Can Use It}
There are several ways to make use of this package. As all features are implemented on the Lua side, you can use these functions on their own. If you do so, please consult the corresponding subsections in the \hyperref{implementation}[sec:implementation] part, because there are some variables that can be adapted to your need.

You can use the following code inside a |\directlua| statement or in a |luacode| environment (or the corresponding thing in your format):
\begin{verbatim}
luatexbase.add_to_callback("pre_linebreak_filter",chickenize,"chickenize the input",1)
\end{verbatim}
Replace |"pre| by |"post| to register into the post linebreak filter. The second argument gives the function name; find a list of available functions below. You can give a label as you like in the third argument, and the last argument gives the order in which the functions in the callback are used. If you have no fancy stuff going on, you can safely use |1|.

If you don't want to mess with the Lua side (but please, try it, you'll learn much!), there is a \LaTeX- as well as a plain\TeX\ interface described in the next section. The commands may not always be on the latest code base – if anything does not work as expected, please tell me and I'll correct it.

\subsection{Commands}
You have a number of commands at your hand, each of which does some manipulation of the input or output. In fact, the code is easy and straightforward, but be careful, especially when combining things. Apply features step by step so your brain won't be damaged …

Some commands have optional arguments that are \emph{only} available for \LaTeX. plain\TeX\ users are mostly capable of finding out how to change things themselfs, but if you are willing to wrap up the code for optional argument processing, don't hesitate sharing it with me ;)
\begin{description}
\item[chickenize] Replaces every word of the input with the word “chicken”. Maybe sometime the replaced word can be changed, but up to now, it's only chicken. To be a bit less static, about every 10\textsuperscript{th} chicken is uppercase. However, the beginning of a sentence is not recognized automatically.\footnote{If you have a nice implementation idea, I'd love to include this!}
\item[uppercasecolor] Makes every uppercase character in the input colored. At the moment, the color is randomized over the full rgb scale, but that will be adjustable once options are well implemented.
\item[randomuclc] Changes every character of the input into its uppercase or lowercase variant. Well, guess what the “random” means …
\item[randomfonts] Changes the font randomly for every character. If no parameters are given, all fonts that have been loaded are used, especially including math fonts.
\item[leetspeak] Translates the input into 1337 speak. If you don't understand that, lern it, n00b.
\item[colorstretch] Inspired by Paul Isambert's code, this command prints boxes instead of lines. The greyness of the first (left-hand) box corresponds to the badness of the line, i.\,e. it is a measure for how much the space between words has been extended to get proper paragraph justification. The second box on the right-hand side shows the amount of stretching/shrinking when font expansion is used. Together the box greyness give you information about how well the overall greyness of the typeset page is. You may specifiy the optional arguments |[(no)keeptext]| to display the text or delete it, also |[(no)colorexpansion]| controls wether or not the font expansion should be evaluated or not.

This functionality is actually the only really usefull implementation of this package …
\end{description}

\subsection{text-Versions}
The functions of this package might be much more useful if applied only to a short sequence of words or single words instead of the whole document or paragraph. Therefore, most of the above-mentioned commands have a |\text|-version that takes an argument. |\textrandomcolor{foo}| results in a colored |foo| while the rest of the document keeps its color. However, to achieve this effect, still the whole node list has to be traversed, so it may slow down your document, even if you use |\textrandomcolor| only once. Fortunately, the effect is very small and mostly negligible.\footnote{On a 500 pages text-only \LaTeX\ document the dilation is on the order of 10\% with |textrandomcolor|, but other manipulations can take much more time. However, you are not supposed to make such long documents with |chickenize|!}

Please don't fool around by mixing a |\text|-version with the non-|\text|-version. If you feel like and are not please with the result, it is up to \emph{you} to provide a stable and working solution.
\DocInput{chickenize.dtx}
\end{document}
%</driver>
%<*tex>
% \fi
%\part{Implementation}
%\label{sec:implementation}
%\section{\TeX\ file}
%    \begin{macrocode}
\input{luatexbase.sty}
\directlua{dofile("chickenize.lua")}

\def\chickenize{
  \directlua{luatexbase.add_to_callback("pre_linebreak_filter",chickenize,"chickenize the input",1)}
}
\def\colorstretch{
  \directlua{luatexbase.add_to_callback("post_linebreak_filter",colorstretch,"show stretch and expansion",1)}
}
\def\leetspeak{
  \directlua{luatexbase.add_to_callback("post_linebreak_filter",leet,"transform input to 1337",1)}
}
\def\randomcolor{
  \directlua{luatexbase.add_to_callback("post_linebreak_filter",randomcolor,"random color",1)}
}
\def\randomfonts{
  \directlua{luatexbase.add_to_callback("post_linebreak_filter",randomfonts,"random fonts",1)}
}
\def\randomuclc{
  \directlua{luatexbase.add_to_callback("pre_linebreak_filter",randomuclc,"randomize uc/lc chars",1)}
}
\def\uppercasecolor{
  \directlua{luatexbase.add_to_callback("post_linebreak_filter",uppercasecolor,"color all uc chars",1)}
}

\newluatexattribute\randcolorattr
\def\textrandomcolor#1{%
\randomcolor%
\setluatexattribute\randcolorattr{42}#1%
\unsetluatexattribute\randcolorattr%
\directlua{randomcolor_onlytext=true}%
\gdef\textrandomcolor#1{%
\setluatexattribute\randcolorattr{42}#1%
\unsetluatexattribute\randcolorattr}
} %% to turn off automatic all-colorizing

%    \end{macrocode}
% \iffalse
%</tex>
%<*package>
% \fi
% \section{Preparation}
% Loading of packages and defition of constants. Will change somewhat when migrating to expl3 (?)
%    \begin{macrocode}
% \iffalse
%<*driver>
%% File: chickenize.dtx by Arno Trautmann, mail: arno dot trautmann at gmx dot de
\iffalse
%</driver>
%<*readme>
This is the README file that should contain some important information. So far I can only tell you to run the lualatex on the file chickenize.dtx to produce the four files
  chickenize.pdf  (documentation)
  chickenize.tex  (low-level commands; plainTeX)
  chickenize.sty  (LaTeX user interface)
  chickenize.lua  (Lua package code)

You need an up-to-date TeX Live (2011, if possible) to use this package.

For any comments or suggestions, contact me:
arno dot trautmann at gmx dot de

Hope you have fun with this!
%</readme>
%<*driver>
\fi
\def\nameofplainTeX{plain}
\ifx\fmtname\nameofplainTeX\else
  \expandafter\begingroup
\fi
\input docstrip.tex
\keepsilent
\let\MetaPrefix\relax
\preamble

EXPERIMENTAL CODE

Do not distribute this file without also distributing the
source files specified above.

Do not distribute a modified version of this file under the same name.

\endpreamble
\let\MetaPrefix\DoubleperCent
\postamble
\endpostamble
\askforoverwritefalse

\generate{\file{chickenize.sty}{\from{chickenize.dtx}{package}}}
\generate{\file{chickenize.tex}{\from{chickenize.dtx}{tex}}}
\def\MetaPrefix{-- }
\def\luapostamble{%
  \MetaPrefix^^J%
  \MetaPrefix\space End of File `\outFileName'.%
}
\def\currentpostamble{\luapostamble}%

\generate{\file{chickenize.lua}{\from{chickenize.dtx}{lua}}}
\nopreamble\nopostamble
\generate{\file{readme.markdown}{\from{chickenize.dtx}{readme}}}


\ifx\fmtname\nameofplainTeX
  \expandafter\endbatchfile
\else
  \expandafter\endgroup
\fi
\ProvidesFile{chickenize.sty}
  [v0.1 chickenize package]
\documentclass[english]{ltxdoc}
\usepackage{
  array,
  babel,
  booktabs,
  fontspec,
  graphicx,
  hyperref,
  tabu
}
\hypersetup{
  colorlinks=true,
  linkcolor=blue,
  filecolor=blue,
  urlcolor=blue
}

\setmainfont{TeX Gyre Pagella}

\title{chickenize}
\author{Arno Trautmann\\ \href{mailto:arno.trautmann@gmx.de}{arno.trautmann@gmx.de}}
\begin{document}
\maketitle
\kern-6cm
{\hfill \parbox{6.3cm}{{\fontspec{TeX Gyre Chorus}{\large »\,}The Monthy Pythons, were they \TeX~users,\\\hspace*{.2em} could have written the chickenize macro.{\large \textit «}}\\
\hspace*{4cm}\small Paul Isambert}\kern-2.5cm
}

\kern4cm
\begin{abstract}
This is the package |chickenize|. It allows you to substitute or change the contents of a Lua\LaTeX\ document\footnote{The code is based on pure Lua\TeX\ features, so don't try to use it with any other \TeX\ flavour.}, but is actually only for fun. Please \emph{never} use any of the functionality of this package for a production document. The following table informs you shortly about your possibilities and provides links to the Lua functions. A \LaTeX\ and plain\TeX\ interface is also offered, see below.\\[2ex]
\begin{tabu}{lX[j]}
\toprule
function & \hfill effect \hfill\, \\
\midrule
\hyperref[sec:chickenize]{chickenize} & replaces every word with “chicken”\\
\hyperref[sec:colorstretch]{colorstretch} & shows grey boxes that depict the badness of a line\\
\hyperref[sec:leetspeak]{leetspeak} & translates every letter into the corresponding 1337 letter\\
\hyperref[sec:randomuclc]{randomuclc} & changes randomly between uppercase and lowercase\\
\hyperref[sec:randomfonts]{randomfonts} & changes the font randomly between every letter\\
\hyperref[sec:randomchars]{randomchars} & randomizes the whole input\\
\hyperref[sec:uppercasecolor]{uppercasecolor} & adds a color to every uppercase letter\\
\bottomrule
\end{tabu}
\vspace*{2ex}
\noindent If you have any suggestions or comments, just drop me a mail, I’ll be happy to get any response!
\end{abstract}
\tableofcontents
\section{How It Works}
We make use of Lua\TeX s callbacks, especially the |pre_linebreak_filter| and the |post_linebreak_filter|. Hooking a function into these, we can nearly arbitrarily change the contents of the document. If the changes should be on the input-side (replacing with |chicken|), one can use the |pre_linebreak_filter|. Hower, changes like inserting color are best made after the linebreak is finalized, so |post_linebreak_filter| is used for such things.

\section{How You Can Use It}
There are several ways to make use of this package. As all features are implemented on the Lua side, you can use these functions on their own. If you do so, please consult the corresponding subsections in the \hyperref{implementation}[sec:implementation] part, because there are some variables that can be adapted to your need.

You can use the following code inside a |\directlua| statement or in a |luacode| environment (or the corresponding thing in your format):
\begin{verbatim}
luatexbase.add_to_callback("pre_linebreak_filter",chickenize,"chickenize the input",1)
\end{verbatim}
Replace |"pre| by |"post| to register into the post linebreak filter. The second argument gives the function name; find a list of available functions below. You can give a label as you like in the third argument, and the last argument gives the order in which the functions in the callback are used. If you have no fancy stuff going on, you can safely use |1|.

If you don't want to mess with the Lua side (but please, try it, you'll learn much!), there is a \LaTeX- as well as a plain\TeX\ interface described in the next section. The commands may not always be on the latest code base – if anything does not work as expected, please tell me and I'll correct it.

\subsection{Commands}
You have a number of commands at your hand, each of which does some manipulation of the input or output. In fact, the code is easy and straightforward, but be careful, especially when combining things. Apply features step by step so your brain won't be damaged …

Some commands have optional arguments that are \emph{only} available for \LaTeX. plain\TeX\ users are mostly capable of finding out how to change things themselfs, but if you are willing to wrap up the code for optional argument processing, don't hesitate sharing it with me ;)
\begin{description}
\item[chickenize] Replaces every word of the input with the word “chicken”. Maybe sometime the replaced word can be changed, but up to now, it's only chicken. To be a bit less static, about every 10\textsuperscript{th} chicken is uppercase. However, the beginning of a sentence is not recognized automatically.\footnote{If you have a nice implementation idea, I'd love to include this!}
\item[uppercasecolor] Makes every uppercase character in the input colored. At the moment, the color is randomized over the full rgb scale, but that will be adjustable once options are well implemented.
\item[randomuclc] Changes every character of the input into its uppercase or lowercase variant. Well, guess what the “random” means …
\item[randomfonts] Changes the font randomly for every character. If no parameters are given, all fonts that have been loaded are used, especially including math fonts.
\item[leetspeak] Translates the input into 1337 speak. If you don't understand that, lern it, n00b.
\item[colorstretch] Inspired by Paul Isambert's code, this command prints boxes instead of lines. The greyness of the first (left-hand) box corresponds to the badness of the line, i.\,e. it is a measure for how much the space between words has been extended to get proper paragraph justification. The second box on the right-hand side shows the amount of stretching/shrinking when font expansion is used. Together the box greyness give you information about how well the overall greyness of the typeset page is. You may specifiy the optional arguments |[(no)keeptext]| to display the text or delete it, also |[(no)colorexpansion]| controls wether or not the font expansion should be evaluated or not.

This functionality is actually the only really usefull implementation of this package …
\end{description}

\subsection{text-Versions}
The functions of this package might be much more useful if applied only to a short sequence of words or single words instead of the whole document or paragraph. Therefore, most of the above-mentioned commands have a |\text|-version that takes an argument. |\textrandomcolor{foo}| results in a colored |foo| while the rest of the document keeps its color. However, to achieve this effect, still the whole node list has to be traversed, so it may slow down your document, even if you use |\textrandomcolor| only once. Fortunately, the effect is very small and mostly negligible.\footnote{On a 500 pages text-only \LaTeX\ document the dilation is on the order of 10\% with |textrandomcolor|, but other manipulations can take much more time. However, you are not supposed to make such long documents with |chickenize|!}

Please don't fool around by mixing a |\text|-version with the non-|\text|-version. If you feel like and are not please with the result, it is up to \emph{you} to provide a stable and working solution.
\DocInput{chickenize.dtx}
\end{document}
%</driver>
%<*tex>
% \fi
%\part{Implementation}
%\label{sec:implementation}
%\section{\TeX\ file}
%    \begin{macrocode}
\input{luatexbase.sty}
\directlua{dofile("chickenize.lua")}

\def\chickenize{
  \directlua{luatexbase.add_to_callback("pre_linebreak_filter",chickenize,"chickenize the input",1)}
}
\def\colorstretch{
  \directlua{luatexbase.add_to_callback("post_linebreak_filter",colorstretch,"show stretch and expansion",1)}
}
\def\leetspeak{
  \directlua{luatexbase.add_to_callback("post_linebreak_filter",leet,"transform input to 1337",1)}
}
\def\randomcolor{
  \directlua{luatexbase.add_to_callback("post_linebreak_filter",randomcolor,"random color",1)}
}
\def\randomfonts{
  \directlua{luatexbase.add_to_callback("post_linebreak_filter",randomfonts,"random fonts",1)}
}
\def\randomuclc{
  \directlua{luatexbase.add_to_callback("pre_linebreak_filter",randomuclc,"randomize uc/lc chars",1)}
}
\def\uppercasecolor{
  \directlua{luatexbase.add_to_callback("post_linebreak_filter",uppercasecolor,"color all uc chars",1)}
}

\newluatexattribute\randcolorattr
\def\textrandomcolor#1{%
\randomcolor%
\setluatexattribute\randcolorattr{42}#1%
\unsetluatexattribute\randcolorattr%
\directlua{randomcolor_onlytext=true}%
\gdef\textrandomcolor#1{%
\setluatexattribute\randcolorattr{42}#1%
\unsetluatexattribute\randcolorattr}
} %% to turn off automatic all-colorizing

%    \end{macrocode}
% \iffalse
%</tex>
%<*package>
% \fi
% \section{Preparation}
% Loading of packages and defition of constants. Will change somewhat when migrating to expl3 (?)
%    \begin{macrocode}
\input{chickenize}
\RequirePackage{
  expl3,
  xkeyval,
  xparse
}
%% So far, no keys are defined. This will change …
\ExplSyntaxOn
\NewDocumentCommand\chickenizesetup{m}{
  \directlua{#1}
}
%    \end{macrocode}
% \section{Definition of User-Level Macros}
%    \begin{macrocode}
\DeclareDocumentCommand\chickenize{}{
  \directlua{luatexbase.add_to_callback("pre_linebreak_filter",chickenize,"chickenize the input",1)}
  %% We want to “chickenize” figures, too. So …
  \DeclareDocumentCommand\includegraphics{O{}m}{
     \fbox{Chicken}  %% actually, I'd love to draw a mp graph showing a chicken …
  }
}
%% specials: the balmerpeak. A tribute to http://xkcd.com/323/.
%%           (most probable only available for \LaTeX)

\ExplSyntaxOff  %% because of the : in the domain …
\NewDocumentCommand\balmerpeak{G{}O{-4cm}}{
  \begin{tikzpicture}
  \hspace*{#2}  %% anyhow necessary to fix centering … strange :(
  \begin{axis}
  [width=10cm,height=7cm,
   xmin=-0.005,xmax=0.28,ymin=-0.05,ymax=1,
   xtick={0,0.02,...,0.27},ytick=\empty,
   /pgf/number format/precision=3,/pgf/number format/fixed,
   tick label style={font=\small},
   label style = {font=\Large},
   xlabel = \fontspec{Punk Nova} BLOOD ALCOHOL CONCENTRATION (\%),
   ylabel = \fontspec{Punk Nova} \rotatebox{-90}{\parbox{3cm}{\center programming\\ skills}}]
    \addplot
      [domain=-0.01:0.27,color=red,samples=250]
      {0.8*exp(-0.5*((x-0.1335)^2)/.00002)+
       0.5*exp(-0.5*((x+0.015)^2)/0.01)
      };
  \end{axis}
  \end{tikzpicture}
}
\ExplSyntaxOn
%    \end{macrocode}
%\iffalse
%</package>
%<*lua>
%\fi
% \section{Lua Module}
% This file contains all the necessary functions, sorted alphabetically, not by sense.
% 
% First, we set up some constants. These are made global so the code can be manipulated on document level, too.
%    \begin{macrocode}
Hhead = node.id("hhead")
RULE = node.id("rule")
GLUE = node.id("glue")
WHAT = node.id("whatsit")
COL = node.subtype("pdf_colorstack")
GLYPH = node.id("glyph")
%    \end{macrocode}
% Now we set up the nodes used for all color things. The nodes are whatsits of subtype |pdf_colorstack|.
%    \begin{macrocode}
color_push = node.new(WHAT,COL)
color_pop = node.new(WHAT,COL)
color_push.stack = 0
color_pop.stack = 0
color_push.cmd = 1
color_pop.cmd = 2

%    \end{macrocode}
% \subsection{chickenize}\label{sec:chickenize}
% The infamous |\chickenize| macro. Substitutes every word of the input with the given string. This can be elaborated arbitrarily, and whenever I feel like, I might add functionality. So far, only the string replaces the word, and even hyphenation is not possible.
%    \begin{macrocode}
chickenstring = "Chicken"

local tbl = font.getfont(font.current())
local space = tbl.parameters.space
local shrink = tbl.parameters.space_shrink
local stretch = tbl.parameters.space_stretch
local match = unicode.utf8.match

function chickenize(head)
  for i in node.traverse_id(37,head) do  --find start of a word
    while ((i.next.id == 37) or (i.next.id == 11) or (i.next.id == 7) or (i.next.id == 0)) do  --find end of a word
      i.next = i.next.next
    end
 
    chicken = {}
    chicken[0] = node.new(37,1)  -- only a dummy for the loop
    for i = 1,string.len(chickenstring) do 
      chicken[i] = node.new(37,1)
      chicken[i].font = font.current()
      chicken[i-1].next = chicken[i]
    end
    
    j = 1
    for s in string.utfvalues(chickenstring) do
    local char = unicode.utf8.char(s)
      chicken[j].char = s
      if match(char,"%s") then 
        chicken[j] = node.new(10)
        chicken[j].spec = node.new(47)
        chicken[j].spec.width = space
        chicken[j].spec.shrink = shrink
        chicken[j].spec.stretch = stretch
      end
      j = j+1
    end

    node.insert_before(head,i,chicken[1])
    chicken[1].next = chicken[2] -- seems to be necessary … to be fixed
    chicken[string.len(chickenstring)].next = i.next
  end

  return head
end
%    \end{macrocode}
% \subsection{leet}\label{sec:leet}
% The |leettable| is the substitution scheme. Just add items if you feel to. Maybe we will differ between a light-weight version and a hardcore 1337.
%    \begin{macrocode}
leettable = {
  [101] = 51, -- e
  [105] = 49, -- i
  [108] = 49, -- l
  [111] = 48, -- o
  [115] = 53, -- s
  [116] = 55, -- t

  [101-32] = 51, -- e
  [105-32] = 49, -- i
  [108-32] = 49, -- l
  [111-32] = 48, -- o
  [115-32] = 53, -- s
  [116-32] = 55, -- t
}
%    \end{macrocode}
% And the function. So simple that I will not write any 
%    \begin{macrocode}
function leet(head)
  for line in node.traverse_id(Hhead,head) do 
    for i in node.traverse_id(GLYPH,line.head) do
      if leettable[i.char] then 
        i.char = leettable[i.char]
      end
    end
  end
  return head
end
%    \end{macrocode}
% \subsection{randomfonts}\label{sec:randomfonts}
% Traverses the output and substitutes fonts randomly. A check is done so that the font number is existing.
% One day, the fonts should be easily given explicitely in terms of |\bf| etc.
%    \begin{macrocode}
randomfontslower = 1
randomfontsupper = 0
% 
function randomfonts(head)
  if (randomfontsupper > 0) then rfub = randomfontsupper else rfub = font.max() end  -- either a user-defined or all currently loaded fonts
  for line in node.traverse_id(Hhead,head) do
    for i in node.traverse_id(GLYPH,line.head) do
      i.font = math.random(randomfontslower,rfub)
    end
  end
  return head
end
%    \end{macrocode}
% \subsection{randomuclc}\label{sec:randomuclc}
% Traverses the input list and changes lowercase/uppercase codes.
%    \begin{macrocode}
uclcratio = 0.5 -- so, this can even be changed!
randomuclc = function(head)
  for i in node.traverse_id(37,head) do
    if math.random() < uclcratio then
      i.char = tex.uccode[i.char] 
    else 
      i.char = tex.lccode[i.char]
	end
  end
  return head
end
%    \end{macrocode}
% \subsection{randomchars}\label{sec:randomchars}
%    \begin{macrocode}
randomchars = function (head)
  for line in node.traverse_id(Hhead,head) do
    for i in node.traverse_id(GLYPH,line.head) do
      i.char = math.floor(math.random()*512)
    end
  end
  return head
end
%    \end{macrocode}
% \subsection{randomcolor}\label{sec:randomcolor}
% Setup of the boolean for grey/color, and boundaries for the colors. rgb space is fully used, but greyscale is only used in a visible range, i.\,e. to 90\% instead of 100\% white.
%    \begin{macrocode}
randomcolor_grey = false
randomcolor_onlytext = false --switch between local and global colorization
-- false means “color everything”
Rgb_lower = 1
rGb_lower = 1
rgB_lower = 1
Rgb_upper = 254
rGb_upper = 254
rgB_upper = 254
grey_lower = 0
grey_upper = 900
%    \end{macrocode}
% This function produces the string needed for the pdf color stack. We need values 0]..[1 for the colors.
%    \begin{macrocode}
function randomcolorstring()
  if randomcolorgrey then
    return (0.001*math.random(grey_lower,grey_upper)).." g"
  else 
    Rgb = math.random(Rgb_lower,Rgb_upper)/255
    rGb = math.random(rGb_lower,rGb_upper)/255
    rgB = math.random(rgB_lower,rgB_upper)/255
    return Rgb..rGb..rgB.." rg"
  end
end
%    \end{macrocode}
% The function that does all the colorizing action. It goes through the whole paragraph and looks at every glyph. If the boolean |randomcolor_onlytext| is set, only glyphs with the set attribute will be colored. Elsewise, all glyphs are taken.
%    \begin{macrocode}
function randomcolor(head)
  for line in node.traverse_id(0,head) do
    for i in node.traverse_id(37,line.head) do
      if not(randomcolor_onlytext) or (node.has_attribute(i,luatexbase.attributes.randcolorattr)) then
        color_push.data = randomcolorstring()  -- color or grey string
        line.head = node.insert_before(line.head,i,node.copy(color_push))
        node.insert_after(line.head,i,node.copy(color_pop))
      end
    end
  end
  return head
end
%    \end{macrocode}
% \subsection{uppercasecolor}\label{sec:uppercasecolor}
% Loop through all the nodes and checking whether it is uppercase. If so (and also for small caps), color it.
%    \begin{macrocode}
uppercasecolor = function (head)
  for line in node.traverse_id(Hhead,head) do 
    for upper in node.traverse_id(GLYPH,line.head) do
      if (((upper.char > 64) and (upper.char < 91)) or
          ((upper.char > 57424) and (upper.char < 57451)))  then  -- for small caps! nice ☺
        color_push.data = randomcolorstring()  -- color or grey string
        line.head = node.insert_before(line.head,upper,node.copy(color_push))
        node.insert_after(line.head,upper,node.copy(color_pop))
      end
    end
  end
  return head
end
%    \end{macrocode}
% \subsubsection{colorstretch}\label{sec:colorstretch}
% This function displays the amount of stretching that has been done for each line of an arbitrary document. A well-typeset document should be equally grey over all lines, which is not always possible.
% 
% The function shows in fact two boxes: The first (left) box shows the badness, i.\,e. the amount of stretching the spaces between words. Too much space results in ligth gray, whereas a too dense line is indicated by a dark grey box. 
%
%The second box is only usefull if microtypographic extensions are used, e.\,g. with the |microtype| package under \LaTeX. The box color then corresponds to the amount of font expansion in the line. This can be greatly used to show the positive effect of font expansion on the badness of a line!
%
% The base structure of the following code is written by Paul Isambert. Thanks for the code and support, Paul!
%
% Two booleans, |keeptext|, and |colorexpansion|, are used to control the behaviour of the function.
%    \begin{macrocode}
keeptext = true
colorexpansion = true
%    \end{macrocode}
% After setting the constants, the function starts. It receives the vertical list of the typeset paragraph as |head|, and loops through all horizontal lists.
% 
% If font expansion should be shown (|colorexpansion == true|), then the first glyph node is determined and its width compared with the width of the unexpanded glyph. This gives a measure for the expansion factor and is translated into a grey scale.
%    \begin{macrocode}
colorstretch = function (head)

  local f = font.getfont(font.current()).characters
  for line in node.traverse_id(Hhead,head) do 
    local rule_bad = node.new(RULE)

	if colorexpansion then  -- if also the font expansion should be shown
      local g = line.head
        while not(g.id == 37) do
         g = g.next
        end
      exp_factor = g.width / f[g.char].width
      exp_color = .5 + (1-exp_factor)*10 .. " g"
      rule_bad.width = 0.5*line.width  -- we need two rules on each line!
    else 
      rule_bad.width = line.width  -- only the space expansion should be shown, only one rule
    end   
%    \end{macrocode}
% Height and depth of the rules are adapted to print a closed grey pattern, so no white interspace is left.
%
% The glue order and sign can be obtained directly and are translated into a grey scale.
%    \begin{macrocode}
    rule_bad.height = tex.baselineskip.width*4/5  -- this should give a quite nice output!
    rule_bad.depth = tex.baselineskip.width*1/5
  
    local glue_ratio = 0
    if line.glue_order == 0 then
      if line.glue_sign == 1 then
        glue_ratio = .5 * math.min(line.glue_set,1)
      else
        glue_ratio = -.5 * math.min(line.glue_set,1)
      end
    end
    color_push.data = .5 + glue_ratio .. " g"
%    \end{macrocode}
% Now, we throw everything together in a way that works. Somehow …
%    \begin{macrocode}
-- set up output    
    local p = line.head

  -- a rule to immitate kerning all the way back
    local kern_back = node.new(RULE)
    kern_back.width = -line.width

  -- if the text should still be displayed, the color and box nodes are inserted additionally 
  -- and the head is set to the color node
    if keeptext then
      line.head = node.insert_before(line.head,line.head,node.copy(color_push)) -- make the color the first thing
    else
      node.flush_list(p)
      line.head = node.copy(color_push)
    end
    node.insert_after(line.head,line.head,rule_bad)  -- then the rule
    node.insert_after(line.head,line.head.next,node.copy(color_pop)) -- and then pop!
    tmpnode =  node.insert_after(line.head,line.head.next.next,kern_back) 

    -- then a rule with the expansion color
    if colorexpansion then  -- if also the stretch/shrink of letters should be shown
      color_push.data = exp_color
      node.insert_after(line.head,tmpnode,node.copy(color_push))
      node.insert_after(line.head,tmpnode.next,node.copy(rule_bad)) 
      node.insert_after(line.head,tmpnode.next.next,node.copy(color_pop))
    end
  end
  return head
end
%    \end{macrocode}
% And that's it :)
%\iffalse 
% </lua>
%\fi
% \section{Known Bugs}\label{bugs}
% There are surely some bugs …
% \begin{description}
% \item ???
% \end{description}
% \section{To Dos}
% Some things that should be implemented but aren’t so far or are very poor at the moment:
% \begin{description}
% \item ?
% \end{description}
% \Finale
% \endinput
\RequirePackage{
  expl3,
  xkeyval,
  xparse
}
%% So far, no keys are defined. This will change …
\ExplSyntaxOn
\NewDocumentCommand\chickenizesetup{m}{
  \directlua{#1}
}
%    \end{macrocode}
% \section{Definition of User-Level Macros}
%    \begin{macrocode}
\DeclareDocumentCommand\chickenize{}{
  \directlua{luatexbase.add_to_callback("pre_linebreak_filter",chickenize,"chickenize the input",1)}
  %% We want to “chickenize” figures, too. So …
  \DeclareDocumentCommand\includegraphics{O{}m}{
     \fbox{Chicken}  %% actually, I'd love to draw a mp graph showing a chicken …
  }
}
%% specials: the balmerpeak. A tribute to http://xkcd.com/323/.
%%           (most probable only available for \LaTeX)

\ExplSyntaxOff  %% because of the : in the domain …
\NewDocumentCommand\balmerpeak{G{}O{-4cm}}{
  \begin{tikzpicture}
  \hspace*{#2}  %% anyhow necessary to fix centering … strange :(
  \begin{axis}
  [width=10cm,height=7cm,
   xmin=-0.005,xmax=0.28,ymin=-0.05,ymax=1,
   xtick={0,0.02,...,0.27},ytick=\empty,
   /pgf/number format/precision=3,/pgf/number format/fixed,
   tick label style={font=\small},
   label style = {font=\Large},
   xlabel = \fontspec{Punk Nova} BLOOD ALCOHOL CONCENTRATION (\%),
   ylabel = \fontspec{Punk Nova} \rotatebox{-90}{\parbox{3cm}{\center programming\\ skills}}]
    \addplot
      [domain=-0.01:0.27,color=red,samples=250]
      {0.8*exp(-0.5*((x-0.1335)^2)/.00002)+
       0.5*exp(-0.5*((x+0.015)^2)/0.01)
      };
  \end{axis}
  \end{tikzpicture}
}
\ExplSyntaxOn
%    \end{macrocode}
%\iffalse
%</package>
%<*lua>
%\fi
% \section{Lua Module}
% This file contains all the necessary functions, sorted alphabetically, not by sense.
% 
% First, we set up some constants. These are made global so the code can be manipulated on document level, too.
%    \begin{macrocode}
Hhead = node.id("hhead")
RULE = node.id("rule")
GLUE = node.id("glue")
WHAT = node.id("whatsit")
COL = node.subtype("pdf_colorstack")
GLYPH = node.id("glyph")
%    \end{macrocode}
% Now we set up the nodes used for all color things. The nodes are whatsits of subtype |pdf_colorstack|.
%    \begin{macrocode}
color_push = node.new(WHAT,COL)
color_pop = node.new(WHAT,COL)
color_push.stack = 0
color_pop.stack = 0
color_push.cmd = 1
color_pop.cmd = 2

%    \end{macrocode}
% \subsection{chickenize}\label{sec:chickenize}
% The infamous |\chickenize| macro. Substitutes every word of the input with the given string. This can be elaborated arbitrarily, and whenever I feel like, I might add functionality. So far, only the string replaces the word, and even hyphenation is not possible.
%    \begin{macrocode}
chickenstring = "Chicken"

local tbl = font.getfont(font.current())
local space = tbl.parameters.space
local shrink = tbl.parameters.space_shrink
local stretch = tbl.parameters.space_stretch
local match = unicode.utf8.match

function chickenize(head)
  for i in node.traverse_id(37,head) do  --find start of a word
    while ((i.next.id == 37) or (i.next.id == 11) or (i.next.id == 7) or (i.next.id == 0)) do  --find end of a word
      i.next = i.next.next
    end
 
    chicken = {}
    chicken[0] = node.new(37,1)  -- only a dummy for the loop
    for i = 1,string.len(chickenstring) do 
      chicken[i] = node.new(37,1)
      chicken[i].font = font.current()
      chicken[i-1].next = chicken[i]
    end
    
    j = 1
    for s in string.utfvalues(chickenstring) do
    local char = unicode.utf8.char(s)
      chicken[j].char = s
      if match(char,"%s") then 
        chicken[j] = node.new(10)
        chicken[j].spec = node.new(47)
        chicken[j].spec.width = space
        chicken[j].spec.shrink = shrink
        chicken[j].spec.stretch = stretch
      end
      j = j+1
    end

    node.insert_before(head,i,chicken[1])
    chicken[1].next = chicken[2] -- seems to be necessary … to be fixed
    chicken[string.len(chickenstring)].next = i.next
  end

  return head
end
%    \end{macrocode}
% \subsection{leet}\label{sec:leet}
% The |leettable| is the substitution scheme. Just add items if you feel to. Maybe we will differ between a light-weight version and a hardcore 1337.
%    \begin{macrocode}
leettable = {
  [101] = 51, -- e
  [105] = 49, -- i
  [108] = 49, -- l
  [111] = 48, -- o
  [115] = 53, -- s
  [116] = 55, -- t

  [101-32] = 51, -- e
  [105-32] = 49, -- i
  [108-32] = 49, -- l
  [111-32] = 48, -- o
  [115-32] = 53, -- s
  [116-32] = 55, -- t
}
%    \end{macrocode}
% And the function. So simple that I will not write any 
%    \begin{macrocode}
function leet(head)
  for line in node.traverse_id(Hhead,head) do 
    for i in node.traverse_id(GLYPH,line.head) do
      if leettable[i.char] then 
        i.char = leettable[i.char]
      end
    end
  end
  return head
end
%    \end{macrocode}
% \subsection{randomfonts}\label{sec:randomfonts}
% Traverses the output and substitutes fonts randomly. A check is done so that the font number is existing.
% One day, the fonts should be easily given explicitely in terms of |\bf| etc.
%    \begin{macrocode}
randomfontslower = 1
randomfontsupper = 0
% 
function randomfonts(head)
  if (randomfontsupper > 0) then rfub = randomfontsupper else rfub = font.max() end  -- either a user-defined or all currently loaded fonts
  for line in node.traverse_id(Hhead,head) do
    for i in node.traverse_id(GLYPH,line.head) do
      i.font = math.random(randomfontslower,rfub)
    end
  end
  return head
end
%    \end{macrocode}
% \subsection{randomuclc}\label{sec:randomuclc}
% Traverses the input list and changes lowercase/uppercase codes.
%    \begin{macrocode}
uclcratio = 0.5 -- so, this can even be changed!
randomuclc = function(head)
  for i in node.traverse_id(37,head) do
    if math.random() < uclcratio then
      i.char = tex.uccode[i.char] 
    else 
      i.char = tex.lccode[i.char]
	end
  end
  return head
end
%    \end{macrocode}
% \subsection{randomchars}\label{sec:randomchars}
%    \begin{macrocode}
randomchars = function (head)
  for line in node.traverse_id(Hhead,head) do
    for i in node.traverse_id(GLYPH,line.head) do
      i.char = math.floor(math.random()*512)
    end
  end
  return head
end
%    \end{macrocode}
% \subsection{randomcolor}\label{sec:randomcolor}
% Setup of the boolean for grey/color, and boundaries for the colors. rgb space is fully used, but greyscale is only used in a visible range, i.\,e. to 90\% instead of 100\% white.
%    \begin{macrocode}
randomcolor_grey = false
randomcolor_onlytext = false --switch between local and global colorization
-- false means “color everything”
Rgb_lower = 1
rGb_lower = 1
rgB_lower = 1
Rgb_upper = 254
rGb_upper = 254
rgB_upper = 254
grey_lower = 0
grey_upper = 900
%    \end{macrocode}
% This function produces the string needed for the pdf color stack. We need values 0]..[1 for the colors.
%    \begin{macrocode}
function randomcolorstring()
  if randomcolorgrey then
    return (0.001*math.random(grey_lower,grey_upper)).." g"
  else 
    Rgb = math.random(Rgb_lower,Rgb_upper)/255
    rGb = math.random(rGb_lower,rGb_upper)/255
    rgB = math.random(rgB_lower,rgB_upper)/255
    return Rgb..rGb..rgB.." rg"
  end
end
%    \end{macrocode}
% The function that does all the colorizing action. It goes through the whole paragraph and looks at every glyph. If the boolean |randomcolor_onlytext| is set, only glyphs with the set attribute will be colored. Elsewise, all glyphs are taken.
%    \begin{macrocode}
function randomcolor(head)
  for line in node.traverse_id(0,head) do
    for i in node.traverse_id(37,line.head) do
      if not(randomcolor_onlytext) or (node.has_attribute(i,luatexbase.attributes.randcolorattr)) then
        color_push.data = randomcolorstring()  -- color or grey string
        line.head = node.insert_before(line.head,i,node.copy(color_push))
        node.insert_after(line.head,i,node.copy(color_pop))
      end
    end
  end
  return head
end
%    \end{macrocode}
% \subsection{uppercasecolor}\label{sec:uppercasecolor}
% Loop through all the nodes and checking whether it is uppercase. If so (and also for small caps), color it.
%    \begin{macrocode}
uppercasecolor = function (head)
  for line in node.traverse_id(Hhead,head) do 
    for upper in node.traverse_id(GLYPH,line.head) do
      if (((upper.char > 64) and (upper.char < 91)) or
          ((upper.char > 57424) and (upper.char < 57451)))  then  -- for small caps! nice ☺
        color_push.data = randomcolorstring()  -- color or grey string
        line.head = node.insert_before(line.head,upper,node.copy(color_push))
        node.insert_after(line.head,upper,node.copy(color_pop))
      end
    end
  end
  return head
end
%    \end{macrocode}
% \subsubsection{colorstretch}\label{sec:colorstretch}
% This function displays the amount of stretching that has been done for each line of an arbitrary document. A well-typeset document should be equally grey over all lines, which is not always possible.
% 
% The function shows in fact two boxes: The first (left) box shows the badness, i.\,e. the amount of stretching the spaces between words. Too much space results in ligth gray, whereas a too dense line is indicated by a dark grey box. 
%
%The second box is only usefull if microtypographic extensions are used, e.\,g. with the |microtype| package under \LaTeX. The box color then corresponds to the amount of font expansion in the line. This can be greatly used to show the positive effect of font expansion on the badness of a line!
%
% The base structure of the following code is written by Paul Isambert. Thanks for the code and support, Paul!
%
% Two booleans, |keeptext|, and |colorexpansion|, are used to control the behaviour of the function.
%    \begin{macrocode}
keeptext = true
colorexpansion = true
%    \end{macrocode}
% After setting the constants, the function starts. It receives the vertical list of the typeset paragraph as |head|, and loops through all horizontal lists.
% 
% If font expansion should be shown (|colorexpansion == true|), then the first glyph node is determined and its width compared with the width of the unexpanded glyph. This gives a measure for the expansion factor and is translated into a grey scale.
%    \begin{macrocode}
colorstretch = function (head)

  local f = font.getfont(font.current()).characters
  for line in node.traverse_id(Hhead,head) do 
    local rule_bad = node.new(RULE)

	if colorexpansion then  -- if also the font expansion should be shown
      local g = line.head
        while not(g.id == 37) do
         g = g.next
        end
      exp_factor = g.width / f[g.char].width
      exp_color = .5 + (1-exp_factor)*10 .. " g"
      rule_bad.width = 0.5*line.width  -- we need two rules on each line!
    else 
      rule_bad.width = line.width  -- only the space expansion should be shown, only one rule
    end   
%    \end{macrocode}
% Height and depth of the rules are adapted to print a closed grey pattern, so no white interspace is left.
%
% The glue order and sign can be obtained directly and are translated into a grey scale.
%    \begin{macrocode}
    rule_bad.height = tex.baselineskip.width*4/5  -- this should give a quite nice output!
    rule_bad.depth = tex.baselineskip.width*1/5
  
    local glue_ratio = 0
    if line.glue_order == 0 then
      if line.glue_sign == 1 then
        glue_ratio = .5 * math.min(line.glue_set,1)
      else
        glue_ratio = -.5 * math.min(line.glue_set,1)
      end
    end
    color_push.data = .5 + glue_ratio .. " g"
%    \end{macrocode}
% Now, we throw everything together in a way that works. Somehow …
%    \begin{macrocode}
-- set up output    
    local p = line.head

  -- a rule to immitate kerning all the way back
    local kern_back = node.new(RULE)
    kern_back.width = -line.width

  -- if the text should still be displayed, the color and box nodes are inserted additionally 
  -- and the head is set to the color node
    if keeptext then
      line.head = node.insert_before(line.head,line.head,node.copy(color_push)) -- make the color the first thing
    else
      node.flush_list(p)
      line.head = node.copy(color_push)
    end
    node.insert_after(line.head,line.head,rule_bad)  -- then the rule
    node.insert_after(line.head,line.head.next,node.copy(color_pop)) -- and then pop!
    tmpnode =  node.insert_after(line.head,line.head.next.next,kern_back) 

    -- then a rule with the expansion color
    if colorexpansion then  -- if also the stretch/shrink of letters should be shown
      color_push.data = exp_color
      node.insert_after(line.head,tmpnode,node.copy(color_push))
      node.insert_after(line.head,tmpnode.next,node.copy(rule_bad)) 
      node.insert_after(line.head,tmpnode.next.next,node.copy(color_pop))
    end
  end
  return head
end
%    \end{macrocode}
% And that's it :)
%\iffalse 
% </lua>
%\fi
% \section{Known Bugs}\label{bugs}
% There are surely some bugs …
% \begin{description}
% \item ???
% \end{description}
% \section{To Dos}
% Some things that should be implemented but aren’t so far or are very poor at the moment:
% \begin{description}
% \item ?
% \end{description}
% \Finale
% \endinput
\RequirePackage{
  expl3,
  xkeyval,
  xparse
}
%% So far, no keys are defined. This will change …
\ExplSyntaxOn
\NewDocumentCommand\chickenizesetup{m}{
  \directlua{#1}
}
%    \end{macrocode}
% \section{Definition of User-Level Macros}
%    \begin{macrocode}
\DeclareDocumentCommand\chickenize{}{
  \directlua{luatexbase.add_to_callback("pre_linebreak_filter",chickenize,"chickenize the input",1)}
  %% We want to “chickenize” figures, too. So …
  \DeclareDocumentCommand\includegraphics{O{}m}{
     \fbox{Chicken}  %% actually, I'd love to draw a mp graph showing a chicken …
  }
}
%% specials: the balmerpeak. A tribute to http://xkcd.com/323/.
%%           (most probable only available for \LaTeX)

\ExplSyntaxOff  %% because of the : in the domain …
\NewDocumentCommand\balmerpeak{G{}O{-4cm}}{
  \begin{tikzpicture}
  \hspace*{#2}  %% anyhow necessary to fix centering … strange :(
  \begin{axis}
  [width=10cm,height=7cm,
   xmin=-0.005,xmax=0.28,ymin=-0.05,ymax=1,
   xtick={0,0.02,...,0.27},ytick=\empty,
   /pgf/number format/precision=3,/pgf/number format/fixed,
   tick label style={font=\small},
   label style = {font=\Large},
   xlabel = \fontspec{Punk Nova} BLOOD ALCOHOL CONCENTRATION (\%),
   ylabel = \fontspec{Punk Nova} \rotatebox{-90}{\parbox{3cm}{\center programming\\ skills}}]
    \addplot
      [domain=-0.01:0.27,color=red,samples=250]
      {0.8*exp(-0.5*((x-0.1335)^2)/.00002)+
       0.5*exp(-0.5*((x+0.015)^2)/0.01)
      };
  \end{axis}
  \end{tikzpicture}
}
\ExplSyntaxOn
%    \end{macrocode}
%\iffalse
%</package>
%<*lua>
%\fi
% \section{Lua Module}
% This file contains all the necessary functions, sorted alphabetically, not by sense.
% 
% First, we set up some constants. These are made global so the code can be manipulated on document level, too.
%    \begin{macrocode}
Hhead = node.id("hhead")
RULE = node.id("rule")
GLUE = node.id("glue")
WHAT = node.id("whatsit")
COL = node.subtype("pdf_colorstack")
GLYPH = node.id("glyph")
%    \end{macrocode}
% Now we set up the nodes used for all color things. The nodes are whatsits of subtype |pdf_colorstack|.
%    \begin{macrocode}
color_push = node.new(WHAT,COL)
color_pop = node.new(WHAT,COL)
color_push.stack = 0
color_pop.stack = 0
color_push.cmd = 1
color_pop.cmd = 2

%    \end{macrocode}
% \subsection{chickenize}\label{sec:chickenize}
% The infamous |\chickenize| macro. Substitutes every word of the input with the given string. This can be elaborated arbitrarily, and whenever I feel like, I might add functionality. So far, only the string replaces the word, and even hyphenation is not possible.
%    \begin{macrocode}
chickenstring = "Chicken"

local tbl = font.getfont(font.current())
local space = tbl.parameters.space
local shrink = tbl.parameters.space_shrink
local stretch = tbl.parameters.space_stretch
local match = unicode.utf8.match

function chickenize(head)
  for i in node.traverse_id(37,head) do  --find start of a word
    while ((i.next.id == 37) or (i.next.id == 11) or (i.next.id == 7) or (i.next.id == 0)) do  --find end of a word
      i.next = i.next.next
    end
 
    chicken = {}
    chicken[0] = node.new(37,1)  -- only a dummy for the loop
    for i = 1,string.len(chickenstring) do 
      chicken[i] = node.new(37,1)
      chicken[i].font = font.current()
      chicken[i-1].next = chicken[i]
    end
    
    j = 1
    for s in string.utfvalues(chickenstring) do
    local char = unicode.utf8.char(s)
      chicken[j].char = s
      if match(char,"%s") then 
        chicken[j] = node.new(10)
        chicken[j].spec = node.new(47)
        chicken[j].spec.width = space
        chicken[j].spec.shrink = shrink
        chicken[j].spec.stretch = stretch
      end
      j = j+1
    end

    node.insert_before(head,i,chicken[1])
    chicken[1].next = chicken[2] -- seems to be necessary … to be fixed
    chicken[string.len(chickenstring)].next = i.next
  end

  return head
end
%    \end{macrocode}
% \subsection{leet}\label{sec:leet}
% The |leettable| is the substitution scheme. Just add items if you feel to. Maybe we will differ between a light-weight version and a hardcore 1337.
%    \begin{macrocode}
leettable = {
  [101] = 51, -- e
  [105] = 49, -- i
  [108] = 49, -- l
  [111] = 48, -- o
  [115] = 53, -- s
  [116] = 55, -- t

  [101-32] = 51, -- e
  [105-32] = 49, -- i
  [108-32] = 49, -- l
  [111-32] = 48, -- o
  [115-32] = 53, -- s
  [116-32] = 55, -- t
}
%    \end{macrocode}
% And the function. So simple that I will not write any 
%    \begin{macrocode}
function leet(head)
  for line in node.traverse_id(Hhead,head) do 
    for i in node.traverse_id(GLYPH,line.head) do
      if leettable[i.char] then 
        i.char = leettable[i.char]
      end
    end
  end
  return head
end
%    \end{macrocode}
% \subsection{randomfonts}\label{sec:randomfonts}
% Traverses the output and substitutes fonts randomly. A check is done so that the font number is existing.
% One day, the fonts should be easily given explicitely in terms of |\bf| etc.
%    \begin{macrocode}
randomfontslower = 1
randomfontsupper = 0
% 
function randomfonts(head)
  if (randomfontsupper > 0) then rfub = randomfontsupper else rfub = font.max() end  -- either a user-defined or all currently loaded fonts
  for line in node.traverse_id(Hhead,head) do
    for i in node.traverse_id(GLYPH,line.head) do
      i.font = math.random(randomfontslower,rfub)
    end
  end
  return head
end
%    \end{macrocode}
% \subsection{randomuclc}\label{sec:randomuclc}
% Traverses the input list and changes lowercase/uppercase codes.
%    \begin{macrocode}
uclcratio = 0.5 -- so, this can even be changed!
randomuclc = function(head)
  for i in node.traverse_id(37,head) do
    if math.random() < uclcratio then
      i.char = tex.uccode[i.char] 
    else 
      i.char = tex.lccode[i.char]
	end
  end
  return head
end
%    \end{macrocode}
% \subsection{randomchars}\label{sec:randomchars}
%    \begin{macrocode}
randomchars = function (head)
  for line in node.traverse_id(Hhead,head) do
    for i in node.traverse_id(GLYPH,line.head) do
      i.char = math.floor(math.random()*512)
    end
  end
  return head
end
%    \end{macrocode}
% \subsection{randomcolor}\label{sec:randomcolor}
% Setup of the boolean for grey/color, and boundaries for the colors. rgb space is fully used, but greyscale is only used in a visible range, i.\,e. to 90\% instead of 100\% white.
%    \begin{macrocode}
randomcolor_grey = false
randomcolor_onlytext = false --switch between local and global colorization
-- false means “color everything”
Rgb_lower = 1
rGb_lower = 1
rgB_lower = 1
Rgb_upper = 254
rGb_upper = 254
rgB_upper = 254
grey_lower = 0
grey_upper = 900
%    \end{macrocode}
% This function produces the string needed for the pdf color stack. We need values 0]..[1 for the colors.
%    \begin{macrocode}
function randomcolorstring()
  if randomcolorgrey then
    return (0.001*math.random(grey_lower,grey_upper)).." g"
  else 
    Rgb = math.random(Rgb_lower,Rgb_upper)/255
    rGb = math.random(rGb_lower,rGb_upper)/255
    rgB = math.random(rgB_lower,rgB_upper)/255
    return Rgb..rGb..rgB.." rg"
  end
end
%    \end{macrocode}
% The function that does all the colorizing action. It goes through the whole paragraph and looks at every glyph. If the boolean |randomcolor_onlytext| is set, only glyphs with the set attribute will be colored. Elsewise, all glyphs are taken.
%    \begin{macrocode}
function randomcolor(head)
  for line in node.traverse_id(0,head) do
    for i in node.traverse_id(37,line.head) do
      if not(randomcolor_onlytext) or (node.has_attribute(i,luatexbase.attributes.randcolorattr)) then
        color_push.data = randomcolorstring()  -- color or grey string
        line.head = node.insert_before(line.head,i,node.copy(color_push))
        node.insert_after(line.head,i,node.copy(color_pop))
      end
    end
  end
  return head
end
%    \end{macrocode}
% \subsection{uppercasecolor}\label{sec:uppercasecolor}
% Loop through all the nodes and checking whether it is uppercase. If so (and also for small caps), color it.
%    \begin{macrocode}
uppercasecolor = function (head)
  for line in node.traverse_id(Hhead,head) do 
    for upper in node.traverse_id(GLYPH,line.head) do
      if (((upper.char > 64) and (upper.char < 91)) or
          ((upper.char > 57424) and (upper.char < 57451)))  then  -- for small caps! nice ☺
        color_push.data = randomcolorstring()  -- color or grey string
        line.head = node.insert_before(line.head,upper,node.copy(color_push))
        node.insert_after(line.head,upper,node.copy(color_pop))
      end
    end
  end
  return head
end
%    \end{macrocode}
% \subsubsection{colorstretch}\label{sec:colorstretch}
% This function displays the amount of stretching that has been done for each line of an arbitrary document. A well-typeset document should be equally grey over all lines, which is not always possible.
% 
% The function shows in fact two boxes: The first (left) box shows the badness, i.\,e. the amount of stretching the spaces between words. Too much space results in ligth gray, whereas a too dense line is indicated by a dark grey box. 
%
%The second box is only usefull if microtypographic extensions are used, e.\,g. with the |microtype| package under \LaTeX. The box color then corresponds to the amount of font expansion in the line. This can be greatly used to show the positive effect of font expansion on the badness of a line!
%
% The base structure of the following code is written by Paul Isambert. Thanks for the code and support, Paul!
%
% Two booleans, |keeptext|, and |colorexpansion|, are used to control the behaviour of the function.
%    \begin{macrocode}
keeptext = true
colorexpansion = true
%    \end{macrocode}
% After setting the constants, the function starts. It receives the vertical list of the typeset paragraph as |head|, and loops through all horizontal lists.
% 
% If font expansion should be shown (|colorexpansion == true|), then the first glyph node is determined and its width compared with the width of the unexpanded glyph. This gives a measure for the expansion factor and is translated into a grey scale.
%    \begin{macrocode}
colorstretch = function (head)

  local f = font.getfont(font.current()).characters
  for line in node.traverse_id(Hhead,head) do 
    local rule_bad = node.new(RULE)

	if colorexpansion then  -- if also the font expansion should be shown
      local g = line.head
        while not(g.id == 37) do
         g = g.next
        end
      exp_factor = g.width / f[g.char].width
      exp_color = .5 + (1-exp_factor)*10 .. " g"
      rule_bad.width = 0.5*line.width  -- we need two rules on each line!
    else 
      rule_bad.width = line.width  -- only the space expansion should be shown, only one rule
    end   
%    \end{macrocode}
% Height and depth of the rules are adapted to print a closed grey pattern, so no white interspace is left.
%
% The glue order and sign can be obtained directly and are translated into a grey scale.
%    \begin{macrocode}
    rule_bad.height = tex.baselineskip.width*4/5  -- this should give a quite nice output!
    rule_bad.depth = tex.baselineskip.width*1/5
  
    local glue_ratio = 0
    if line.glue_order == 0 then
      if line.glue_sign == 1 then
        glue_ratio = .5 * math.min(line.glue_set,1)
      else
        glue_ratio = -.5 * math.min(line.glue_set,1)
      end
    end
    color_push.data = .5 + glue_ratio .. " g"
%    \end{macrocode}
% Now, we throw everything together in a way that works. Somehow …
%    \begin{macrocode}
-- set up output    
    local p = line.head

  -- a rule to immitate kerning all the way back
    local kern_back = node.new(RULE)
    kern_back.width = -line.width

  -- if the text should still be displayed, the color and box nodes are inserted additionally 
  -- and the head is set to the color node
    if keeptext then
      line.head = node.insert_before(line.head,line.head,node.copy(color_push)) -- make the color the first thing
    else
      node.flush_list(p)
      line.head = node.copy(color_push)
    end
    node.insert_after(line.head,line.head,rule_bad)  -- then the rule
    node.insert_after(line.head,line.head.next,node.copy(color_pop)) -- and then pop!
    tmpnode =  node.insert_after(line.head,line.head.next.next,kern_back) 

    -- then a rule with the expansion color
    if colorexpansion then  -- if also the stretch/shrink of letters should be shown
      color_push.data = exp_color
      node.insert_after(line.head,tmpnode,node.copy(color_push))
      node.insert_after(line.head,tmpnode.next,node.copy(rule_bad)) 
      node.insert_after(line.head,tmpnode.next.next,node.copy(color_pop))
    end
  end
  return head
end
%    \end{macrocode}
% And that's it :)
%\iffalse 
% </lua>
%\fi
% \section{Known Bugs}\label{bugs}
% There are surely some bugs …
% \begin{description}
% \item ???
% \end{description}
% \section{To Dos}
% Some things that should be implemented but aren’t so far or are very poor at the moment:
% \begin{description}
% \item ?
% \end{description}
% \Finale
% \endinput
\RequirePackage{
  expl3,
  xkeyval,
  xparse
}
%% So far, no keys are defined. This will change …
\ExplSyntaxOn
\NewDocumentCommand\chickenizesetup{m}{
  \directlua{#1}
}
%    \end{macrocode}
% \section{Definition of User-Level Macros}
%    \begin{macrocode}
\DeclareDocumentCommand\chickenize{}{
  \directlua{luatexbase.add_to_callback("pre_linebreak_filter",chickenize,"chickenize the input",1)}
  %% We want to “chickenize” figures, too. So …
  \DeclareDocumentCommand\includegraphics{O{}m}{
     \fbox{Chicken}  %% actually, I'd love to draw a mp graph showing a chicken …
  }
}
%% specials: the balmerpeak. A tribute to http://xkcd.com/323/.
%%           (most probable only available for \LaTeX)

\ExplSyntaxOff  %% because of the : in the domain …
\NewDocumentCommand\balmerpeak{G{}O{-4cm}}{
  \begin{tikzpicture}
  \hspace*{#2}  %% anyhow necessary to fix centering … strange :(
  \begin{axis}
  [width=10cm,height=7cm,
   xmin=-0.005,xmax=0.28,ymin=-0.05,ymax=1,
   xtick={0,0.02,...,0.27},ytick=\empty,
   /pgf/number format/precision=3,/pgf/number format/fixed,
   tick label style={font=\small},
   label style = {font=\Large},
   xlabel = \fontspec{Punk Nova} BLOOD ALCOHOL CONCENTRATION (\%),
   ylabel = \fontspec{Punk Nova} \rotatebox{-90}{\parbox{3cm}{\center programming\\ skills}}]
    \addplot
      [domain=-0.01:0.27,color=red,samples=250]
      {0.8*exp(-0.5*((x-0.1335)^2)/.00002)+
       0.5*exp(-0.5*((x+0.015)^2)/0.01)
      };
  \end{axis}
  \end{tikzpicture}
}
\ExplSyntaxOn
%    \end{macrocode}
%\iffalse
%</package>
%<*lua>
%\fi
% \section{Lua Module}
% This file contains all the necessary functions, sorted alphabetically, not by sense.
% 
% First, we set up some constants. These are made global so the code can be manipulated on document level, too.
%    \begin{macrocode}
Hhead = node.id("hhead")
RULE = node.id("rule")
GLUE = node.id("glue")
WHAT = node.id("whatsit")
COL = node.subtype("pdf_colorstack")
GLYPH = node.id("glyph")
%    \end{macrocode}
% Now we set up the nodes used for all color things. The nodes are whatsits of subtype |pdf_colorstack|.
%    \begin{macrocode}
color_push = node.new(WHAT,COL)
color_pop = node.new(WHAT,COL)
color_push.stack = 0
color_pop.stack = 0
color_push.cmd = 1
color_pop.cmd = 2

%    \end{macrocode}
% \subsection{chickenize}\label{sec:chickenize}
% The infamous |\chickenize| macro. Substitutes every word of the input with the given string. This can be elaborated arbitrarily, and whenever I feel like, I might add functionality. So far, only the string replaces the word, and even hyphenation is not possible.
%    \begin{macrocode}
chickenstring = "Chicken"

local tbl = font.getfont(font.current())
local space = tbl.parameters.space
local shrink = tbl.parameters.space_shrink
local stretch = tbl.parameters.space_stretch
local match = unicode.utf8.match

function chickenize(head)
  for i in node.traverse_id(37,head) do  --find start of a word
    while ((i.next.id == 37) or (i.next.id == 11) or (i.next.id == 7) or (i.next.id == 0)) do  --find end of a word
      i.next = i.next.next
    end
 
    chicken = {}
    chicken[0] = node.new(37,1)  -- only a dummy for the loop
    for i = 1,string.len(chickenstring) do 
      chicken[i] = node.new(37,1)
      chicken[i].font = font.current()
      chicken[i-1].next = chicken[i]
    end
    
    j = 1
    for s in string.utfvalues(chickenstring) do
    local char = unicode.utf8.char(s)
      chicken[j].char = s
      if match(char,"%s") then 
        chicken[j] = node.new(10)
        chicken[j].spec = node.new(47)
        chicken[j].spec.width = space
        chicken[j].spec.shrink = shrink
        chicken[j].spec.stretch = stretch
      end
      j = j+1
    end

    node.insert_before(head,i,chicken[1])
    chicken[1].next = chicken[2] -- seems to be necessary … to be fixed
    chicken[string.len(chickenstring)].next = i.next
  end

  return head
end
%    \end{macrocode}
% \subsection{leet}\label{sec:leet}
% The |leettable| is the substitution scheme. Just add items if you feel to. Maybe we will differ between a light-weight version and a hardcore 1337.
%    \begin{macrocode}
leettable = {
  [101] = 51, -- e
  [105] = 49, -- i
  [108] = 49, -- l
  [111] = 48, -- o
  [115] = 53, -- s
  [116] = 55, -- t

  [101-32] = 51, -- e
  [105-32] = 49, -- i
  [108-32] = 49, -- l
  [111-32] = 48, -- o
  [115-32] = 53, -- s
  [116-32] = 55, -- t
}
%    \end{macrocode}
% And the function. So simple that I will not write any 
%    \begin{macrocode}
function leet(head)
  for line in node.traverse_id(Hhead,head) do 
    for i in node.traverse_id(GLYPH,line.head) do
      if leettable[i.char] then 
        i.char = leettable[i.char]
      end
    end
  end
  return head
end
%    \end{macrocode}
% \subsection{randomfonts}\label{sec:randomfonts}
% Traverses the output and substitutes fonts randomly. A check is done so that the font number is existing.
% One day, the fonts should be easily given explicitely in terms of |\bf| etc.
%    \begin{macrocode}
randomfontslower = 1
randomfontsupper = 0
% 
function randomfonts(head)
  if (randomfontsupper > 0) then rfub = randomfontsupper else rfub = font.max() end  -- either a user-defined or all currently loaded fonts
  for line in node.traverse_id(Hhead,head) do
    for i in node.traverse_id(GLYPH,line.head) do
      i.font = math.random(randomfontslower,rfub)
    end
  end
  return head
end
%    \end{macrocode}
% \subsection{randomuclc}\label{sec:randomuclc}
% Traverses the input list and changes lowercase/uppercase codes.
%    \begin{macrocode}
uclcratio = 0.5 -- so, this can even be changed!
randomuclc = function(head)
  for i in node.traverse_id(37,head) do
    if math.random() < uclcratio then
      i.char = tex.uccode[i.char] 
    else 
      i.char = tex.lccode[i.char]
	end
  end
  return head
end
%    \end{macrocode}
% \subsection{randomchars}\label{sec:randomchars}
%    \begin{macrocode}
randomchars = function (head)
  for line in node.traverse_id(Hhead,head) do
    for i in node.traverse_id(GLYPH,line.head) do
      i.char = math.floor(math.random()*512)
    end
  end
  return head
end
%    \end{macrocode}
% \subsection{randomcolor}\label{sec:randomcolor}
% Setup of the boolean for grey/color, and boundaries for the colors. rgb space is fully used, but greyscale is only used in a visible range, i.\,e. to 90\% instead of 100\% white.
%    \begin{macrocode}
randomcolor_grey = false
randomcolor_onlytext = false --switch between local and global colorization
-- false means “color everything”
Rgb_lower = 1
rGb_lower = 1
rgB_lower = 1
Rgb_upper = 254
rGb_upper = 254
rgB_upper = 254
grey_lower = 0
grey_upper = 900
%    \end{macrocode}
% This function produces the string needed for the pdf color stack. We need values 0]..[1 for the colors.
%    \begin{macrocode}
function randomcolorstring()
  if randomcolor_grey then
    return (0.001*math.random(grey_lower,grey_upper)).." g"
  else 
    Rgb = math.random(Rgb_lower,Rgb_upper)/255
    rGb = math.random(rGb_lower,rGb_upper)/255
    rgB = math.random(rgB_lower,rgB_upper)/255
    return Rgb..rGb..rgB.." rg"
  end
end
%    \end{macrocode}
% The function that does all the colorizing action. It goes through the whole paragraph and looks at every glyph. If the boolean |randomcolor_onlytext| is set, only glyphs with the set attribute will be colored. Elsewise, all glyphs are taken.
%    \begin{macrocode}
function randomcolor(head)
  for line in node.traverse_id(0,head) do
    for i in node.traverse_id(37,line.head) do
      if not(randomcolor_onlytext) or (node.has_attribute(i,luatexbase.attributes.randcolorattr)) then
        color_push.data = randomcolorstring()  -- color or grey string
        line.head = node.insert_before(line.head,i,node.copy(color_push))
        node.insert_after(line.head,i,node.copy(color_pop))
      end
    end
  end
  return head
end
%    \end{macrocode}
% \subsection{uppercasecolor}\label{sec:uppercasecolor}
% Loop through all the nodes and checking whether it is uppercase. If so (and also for small caps), color it.
%    \begin{macrocode}
uppercasecolor = function (head)
  for line in node.traverse_id(Hhead,head) do 
    for upper in node.traverse_id(GLYPH,line.head) do
      if (((upper.char > 64) and (upper.char < 91)) or
          ((upper.char > 57424) and (upper.char < 57451)))  then  -- for small caps! nice ☺
        color_push.data = randomcolorstring()  -- color or grey string
        line.head = node.insert_before(line.head,upper,node.copy(color_push))
        node.insert_after(line.head,upper,node.copy(color_pop))
      end
    end
  end
  return head
end
%    \end{macrocode}
% \subsubsection{colorstretch}\label{sec:colorstretch}
% This function displays the amount of stretching that has been done for each line of an arbitrary document. A well-typeset document should be equally grey over all lines, which is not always possible.
% 
% The function shows in fact two boxes: The first (left) box shows the badness, i.\,e. the amount of stretching the spaces between words. Too much space results in ligth gray, whereas a too dense line is indicated by a dark grey box. 
%
%The second box is only usefull if microtypographic extensions are used, e.\,g. with the |microtype| package under \LaTeX. The box color then corresponds to the amount of font expansion in the line. This can be greatly used to show the positive effect of font expansion on the badness of a line!
%
% The base structure of the following code is written by Paul Isambert. Thanks for the code and support, Paul!
%
% Two booleans, |keeptext|, and |colorexpansion|, are used to control the behaviour of the function.
%    \begin{macrocode}
keeptext = true
colorexpansion = true
%    \end{macrocode}
% After setting the constants, the function starts. It receives the vertical list of the typeset paragraph as |head|, and loops through all horizontal lists.
% 
% If font expansion should be shown (|colorexpansion == true|), then the first glyph node is determined and its width compared with the width of the unexpanded glyph. This gives a measure for the expansion factor and is translated into a grey scale.
%    \begin{macrocode}
colorstretch = function (head)

  local f = font.getfont(font.current()).characters
  for line in node.traverse_id(Hhead,head) do 
    local rule_bad = node.new(RULE)

	if colorexpansion then  -- if also the font expansion should be shown
      local g = line.head
        while not(g.id == 37) do
         g = g.next
        end
      exp_factor = g.width / f[g.char].width
      exp_color = .5 + (1-exp_factor)*10 .. " g"
      rule_bad.width = 0.5*line.width  -- we need two rules on each line!
    else 
      rule_bad.width = line.width  -- only the space expansion should be shown, only one rule
    end   
%    \end{macrocode}
% Height and depth of the rules are adapted to print a closed grey pattern, so no white interspace is left.
%
% The glue order and sign can be obtained directly and are translated into a grey scale.
%    \begin{macrocode}
    rule_bad.height = tex.baselineskip.width*4/5  -- this should give a quite nice output!
    rule_bad.depth = tex.baselineskip.width*1/5
  
    local glue_ratio = 0
    if line.glue_order == 0 then
      if line.glue_sign == 1 then
        glue_ratio = .5 * math.min(line.glue_set,1)
      else
        glue_ratio = -.5 * math.min(line.glue_set,1)
      end
    end
    color_push.data = .5 + glue_ratio .. " g"
%    \end{macrocode}
% Now, we throw everything together in a way that works. Somehow …
%    \begin{macrocode}
-- set up output    
    local p = line.head

  -- a rule to immitate kerning all the way back
    local kern_back = node.new(RULE)
    kern_back.width = -line.width

  -- if the text should still be displayed, the color and box nodes are inserted additionally 
  -- and the head is set to the color node
    if keeptext then
      line.head = node.insert_before(line.head,line.head,node.copy(color_push)) -- make the color the first thing
    else
      node.flush_list(p)
      line.head = node.copy(color_push)
    end
    node.insert_after(line.head,line.head,rule_bad)  -- then the rule
    node.insert_after(line.head,line.head.next,node.copy(color_pop)) -- and then pop!
    tmpnode =  node.insert_after(line.head,line.head.next.next,kern_back) 

    -- then a rule with the expansion color
    if colorexpansion then  -- if also the stretch/shrink of letters should be shown
      color_push.data = exp_color
      node.insert_after(line.head,tmpnode,node.copy(color_push))
      node.insert_after(line.head,tmpnode.next,node.copy(rule_bad)) 
      node.insert_after(line.head,tmpnode.next.next,node.copy(color_pop))
    end
  end
  return head
end
%    \end{macrocode}
% And that's it!\qquad {\fontsize{40}{0}\raisebox{-2ex}{\fontspec{XITS}☺}}
%\iffalse 
% </lua>
%\fi
% \newpage
% \section{Known Bugs}\label{bugs}
% There are surely some bugs …
% \begin{description}
% \item 
% \end{description}
% \section{To Dos}
% Some things that should be implemented but aren’t so far or are very poor at the moment:
% \begin{description}
% \item ?
% \end{description}
% \Finale
% \endinput